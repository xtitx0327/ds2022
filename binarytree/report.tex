\documentclass[lang=cn,12pt]{elegantbook}

\title{二叉树和节点的逻辑设计}
\subtitle{2022 - 2023 秋冬\quad 数据结构与算法}

\author{王笑同}
\institute{数学与应用数学(强基计划)2101}
\date{\today}
%\bioinfo{自定义}{信息}

%\extrainfo{不要以为抹消过去,重新来过,即可发生什么改变。—— 比企谷八幡}

\setcounter{tocdepth}{3}

\logo{logo-blue.png}
\cover{cover1.jpg}

% 本文档命令
\usepackage{array}
\usepackage{framed}
\newcommand{\ccr}[1]{\makecell{{\color{#1}\rule{1cm}{1cm}}}}

\usepackage{float}

% 修改标题页的橙色带
% \definecolor{customcolor}{RGB}{32,178,170}
% \colorlet{coverlinecolor}{customcolor}

\begin{document}

\maketitle

\mainmatter

截至本次作业发布时,我们已经在课上学习了几种数据结构:单链表、双链表、二叉树、二叉搜索树、平衡二叉树. 而这些数据结构又可以进一步概括为二叉树和链表两种. 若将这二者进一步抽象,又可以得到它们的共性:
\begin{framed}
  \begin{itemize}
    \item 数据存放的单位为“结点”;
    \item 相邻“结点”之间可以互相遍历(或至少可以单向遍历).
  \end{itemize}
\end{framed}

由此可见,二叉树和链表本质都可看作\text{树},甚至是更一般的图. 对于一张图 $G=(V,E)$,其每一个结点可以设计成抽象的基类 \verb|Node|,其应该包含以下成员:
\begin{framed}
  \begin{enumerate}
    \item 数据域 \verb|data|,代表结点存放的数据;
    \item 邻接结点表 \verb|neighbor|,依次存储每一个与该结点直接相连的结点的地址.
  \end{enumerate}
\end{framed}

在其基础上,我们可以派生出适用于链表的结点类(模板) \verb|ListNode|,甚至不用添加额外的成员,只需用 \verb|neighbor[0]| 和 \verb|neighbor[1]| 分别代表结点的前驱和后继即可. 类似地,可以派生出二叉树结点类 \verb|BinaryTreeNode|,用 \verb|neighbor| 的前三个元素依次代表父亲、左儿子、右儿子的结点地址. 利用 \verb|BinaryTreeNode|,可以派生出平衡树的结点类 \verb|AVLNode|,只需额外添加公有成员“子树的高度” \verb|height|,照这样的思路,还可以派生出一般的树和图的结点类.

进一步,我们可以设计二叉树类 \verb|BinaryTree|,其中可以包含以下成员:

\begin{framed}
  \textbf{私有成员:}

  \begin{enumerate}
    \item 根节点 \verb|root|;
    \item ……
  \end{enumerate}

  \textbf{公有成员:}

  \begin{enumerate}
    \item 为指定结点添加左右儿子的函数 \verb|inserLeftChild()| 和 \verb|insertRightChild()|;
    \item 进行前/中/后序遍历,输出树中元素;
    \item ……
  \end{enumerate}
\end{framed}

在 \verb|BinaryTree| 的基础上,可以派生出二叉搜索树类 \verb|BinarySearchTree| 和 AVL 平衡树类 \verb|AVLTree|,这里仅以后者为例,其包含的成员除基类中成员外,还应该有:

\begin{framed}
  \textbf{私有成员:}

  \begin{enumerate}
    \item 旋转函数 \verb|rotate()|;
    \item ……
  \end{enumerate}

  \textbf{公有成员:}

  \begin{enumerate}
    \item 向树中插入节点的函数 \verb|insert()|;
    \item 从树中删除结点的函数 \verb|remove()|;
    \item 将树置为空树的函数 \verb|makeEmpty()|;
    \item 判断树是否为空的函数 \verb|isEmpty()|;
    \item 升序输出二叉树中所有元素的函数 \verb|printTree()|;
    \item ……
  \end{enumerate}
\end{framed}

也可以在二叉树的基础上实现一般的树和图的类模板,具体实现不再写出,其思路是完全类似的. 另外,关于以上函数的更详细解释,可参考头文件代码或 latex 目录下的 doxygen 文档.

\end{document}
